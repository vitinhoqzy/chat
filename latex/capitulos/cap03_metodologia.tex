\chapter{Metodologia}
\label{ch:identificador}
	\begin{resumocapitulo}
		O passo a passo da criação
	\end{resumocapitulo}

	\section{Desenvolvimento}	
Desenvolvemos o backend utilizando JavaScript e WebSocket para proporcionar uma experiência de chat em tempo real.

Inicialmente, implementamos um código que executa e atualiza dinamicamente as funcionalidades à medida que são desenvolvidas. Em seguida, elaboramos toda a estrutura HTML e CSS, começando pela tela inicial. No processo de estilização, o CSS é utilizado para criar uma apresentação visual coesa e atraente.

Por meio do JavaScript, implementamos a lógica necessária para distinguir as mensagens enviadas pelos usuários. Essa lógica garante que as mensagens sejam exibidas à direita quando enviadas pelo usuário local e à esquerda quando provenientes de outros usuários, cada uma com sua própria formatação visual definida previamente no CSS.

	\section{Desenvolvimento}

Durante todo o processo de desenvolvimento, utilizamos o Visual Studio Code , uma plataforma que se destaca pela praticidade e eficiência que oferece.

O design do chat foi meticulosamente planejado para proporcionar uma experiência mais limpa e objetiva aos usuários, através da utilização do modo escuro. Esta abordagem visa promover uma interação mais focada e agradável.

Além disso, o desenvolvimento foi realizado no ambiente Windows 10, com o auxílio da extensão Live Server, uma ferramenta integrada ao Visual Studio Code. O Live Server permite visualizar em tempo real as modificações realizadas no código, facilitando significativamente o processo de desenvolvimento.

 Visual Studio Code - https://code.visualstudio.com/
 